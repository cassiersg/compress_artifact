\documentclass{scrartcl}

\usepackage{tikz-timing}
\usepackage{graphicx}
\usepackage{tikz}
\usepackage{circuitikz}
\usepackage{xcolor}
\usetikzlibrary{shapes.symbols}

\usepackage{listings}
\lstset{
  basicstyle=\ttfamily\footnotesize,
  mathescape
}

\input{macros}
\input{tikzset_cfg.tex}
\input{tikz_color}

\begin{document}


\subsection{Architecture of the $\modAESdpKey$ module}

\begin{figure}
    \centering
    \resizebox{\textwidth}{!}{
        \begin{tikzpicture}
            \input{tikz_dpKey}
        \end{tikzpicture}
    }
    \caption{Global architecture of the module $\modAESdpKey$. The value held by the DFF at index $i$ is depicted by the signal $\dpKeyDFF{i}$ in the HDL.}
    \label{fig:aes_dpKey}
\end{figure}

The module $\modAESdpKey$ is shown in Figure~\ref{fig:aes_dpKey}. It is
organized as a shift register where each register unit holds a masked byte of
the key. The module is split in 4~independent parts, each taking care of the
key scheduling operation on a single row. The sharing of the 128-bit key is
routed from the input with the control signal $\dpKeyCtrlInit$. Two relevant
bytes ordering are depicted on the Figure and both refers to the byte index in
the unmasked key. First, the number on the top depict the byte ordering at the
beginning of a round when the key addition occurs. Second, the bottom number
(between parentheses) depict the byte ordering when a fresh execution starts,
at the last cycle of a round or when the $\SB$ layer results of the key
scheduling are fetch back from the S-boxes, as detailed next. In practice, the
second ordering corresponds to the first one with a rotation of 1 column to the
left.

Concretely, the key scheduling starts by sending the rightmost column of the key
(i.e., then byte indexes 12, 13, 14 and 15) to the S-boxes. The $\texttt{RotWord}$
operation is performed by the routing that sends the key bytes to the S-boxes.
Once computed, the result of the $\SB$ layer is routed back to the core through
the MUX controlled by the signal $\dpKeyCtrlRouteFromSB$. At the same time,
the round constant is applied and the first column (i.e., byte indexes 0,1,2
and 3) of the new key is computed by adding its value to the column coming back
from the S-boxes.  The remaining three columns (i.e., byte indexes [4,5,6,7],
[8,9,10,11] and [12,13,14,15] are then updated sequentially by XORing each
bytes with the value of the last byte updated in the same row. The signal
$\dpKeyCtrlLoop$ is used to make the key shares loop across the key pipeline.
This is required to keep the key material after the $\AK$ operations while the
$\SB$ results of the key scheduling is still under computation. 

\subsection{Internal operation}

Let us first introduce notations for the intermediate states in the AES algorithm with
pseudo-code in Figure~\ref{fig:code_round} and Figure~\ref{fig:code_key}.
Each variable denotes a state or subkey byte at a given step of the algorithm.
In particular, the plaintext (resp. key, ciphertext) byte at index $0\leq i<16$
is denoted \pP{i} (resp. $\pK{i}$, $\pCt{i}$), and the value $\pS{i}{r}$ (resp.
$\pRK{i}{r}$) denotes the byte at index $i$ of the state (resp. round key)
starting the $r$-th round.
When no index is given, the full 128-bit state is considered instead.

\begin{figure}
    \begin{lstlisting}[frame=single]
    %%% First key addition
    for $0\leq i <16$ do
        $\pS{i}{0} = \pP{i} \oplus \pK{i}$;
    done
    
    %%% Perform the rounds
    for $0\leq r < 9$ do 
        % Operation for a single round
        $\pSR{}{r} = \SR(\pS{}{r})$;
        $\pSB{}{r} = \SB(\pSR{}{r})$;
        $\pMC{}{r} = \MC(\pSB{}{r})$;
        $\pAK{}{r} = \AK(\pMC{}{r},\pRK{}{r})$;
        $\pS{}{r+1} = \pAK{}{r}$;
    done
    
    %%% Last round
    $\pSR{}{9}=\SR(\pS{}{9})$;
    $\pSB{}{9}=\SB(\pSR{}{9})$;
    $\pAK{}{9}=\AK(\pSB{}{9})$;
    $\pCt{} = \pAK{}{9}$;
    \end{lstlisting}
    \caption{Pseudo-code of the AES encryption.}
    \label{fig:code_round}
\end{figure}


\begin{figure}
    \begin{lstlisting}[frame=single]
    %%% Key evolution for each round key 
    for $0\leq r < 10$ do
        % Fetch value on which operate
        if $r==0$ then
            $t^r = \pK{}$; 
        else 
            $t^r = \pRK{}{r-1}$;
        end

        % Perform the last column rotation
        $[\pR{0}{r},\pR{1}{r},\pR{2}{r},\pR{3}{r}] = [t_{13}^{r},t_{14}^{r},t_{15}^{r},t_{12}^{r}]$; 

        % Perform SubWord on the rotated column
        $[\pRSB{0}{r},\pRSB{1}{r},\pRSB{2}{r},\pRSB{3}{r}] = [\SW{\pR{0}{r}},\SW{\pR{1}{r}},\SW{\pR{2}{r}},\SW{\pR{3}{r}}]$

        % Compute the first column of the next round key
        $\pRK{0}{r} = \pRSB{0}{r} \oplus t_{0}^{r} \oplus \RCON{r}$;
        $\pRK{1}{r} = \pRSB{1}{r} \oplus t_{1}^{r}$;
        $\pRK{2}{r} = \pRSB{2}{r} \oplus t_{2}^{r}$;
        $\pRK{3}{r} = \pRSB{3}{r} \oplus t_{3}^{r}$;

        % Generate the three remaining columns
        for $1\leq i <4$ do
            for $0\leq j <4$ do
                $\pRK{4i+j}{r} = \pRK{4(i-1)+j}{r} \oplus t_{4i+j}^{r}$;
            done
        done
    done
    \end{lstlisting}
    \caption{Pseudo-code for the AES key evolution.}
    \label{fig:code_key}
\end{figure}

\begin{figure}
    \centering
    \input{figures/tikz_time_pipe_sbox}
    \caption{Data going into / coming from the S-boxes during a round.}
    \label{fig:pipe_sbox}
\end{figure}

\begin{figure}
    \centering
    \input{figures/tikz_time_pipe_dpkey}
    \caption{Data going into / coming from the key scheduling datapath during a round.}
    \label{fig:pipe_dpkey}
\end{figure}

\begin{figure}
    \centering
    \input{figures/tikz_time_pipe_dpstate}
    \caption{Data going into / coming from the round function datapath during a round.}
    \label{fig:pipe_dpstate}
\end{figure}

\begin{figure}
    \centering
    \input{figures/tikz_time_first_round} 
    \caption{Data routing when a new execution starts.}
    \label{fig:time_first_round}
\end{figure}

Using these notations, Figures~\ref{fig:pipe_sbox}, \ref{fig:pipe_dpkey}
and~\ref{fig:pipe_dpstate} describe the evolution of the AES states stored in
the architecture over the computation of one round.
Next, Figures~\ref{fig:time_first_round}, \ref{fig:time_regime}
and~\ref{fig:time_last_round} depict the control signals that drive the
datapath for the first round, middle rounds, and last round.
In particular, for the first round (Figure~\ref{fig:time_first_round}), the
data is fetched by the module when the signal $\portAESInValid$ is asserted if
the core is not busy, there is no ciphertext stored in the core and randomness
is available.
At the next clock cycle, the
internal FSM counters $\timeCnrRound$ and $\timeCnrCycle$ are reset and the
execution begins. The round function and the key scheduling algorithm are
executed in parallel by interleaving the S-boxes usage appropriately. In
particular, the first cycle of the execution is used to start the key
scheduling algorithm by asserting $\AESsboxFeedKey$ and $\AESsboxValidIn$.
During this cycle, the module $\modAESdpKey$ is enabled and the
$\dpKeyCtrlLoop$ (rotating then the columns), while the module $\modAESdpState$
is disabled.

\begin{figure}
    \centering
    \input{figures/tikz_time_regime} 
    \caption{In regime data routing.}
    \label{fig:time_regime}
\end{figure}

Then, the core enters into a nominal regime that computes a round in 8~cycles,
as depicted in Figure~\ref{fig:time_regime}.
A typical round starts with 4~clock cycles during which data is read from the
state registers, XORed with the subkey and fed to the S-boxes, which performs
the $\AK$, $\SR$ and $\SB$ layers for the full state (one column per cycle).
During these cycles, $\AESsboxValidIn$ is asserted and data (state and subkey)
loops over the shift registers. An exception occurs at the fourth cycle (i.e., when $\timeCnrCycle=3$): at this cycle, the S-boxes
output the column of the new subkey value, which is processed by deasserting $\dpKeyCtrlLoop$.  
Next, during the last 4~cycles of a round, the S-boxes output the 4~columns of
the state, on which the $\MC$ layer is directly applied, and the result is
stored in the state registers. At the same time, the subkey update is
finalized, such that a new subkey is ready at the last cycle of a round (i.e.,
$\timeCnrCycle=7$). During this last cycle, the next key schedule round is
started, and a new state round starts at the following cycle. 

\begin{figure}
    \centering
    \input{figures/tikz_time_last_round} 
    \caption{Data routing during last rounds.}
    \label{fig:time_last_round}
\end{figure}

%\todo{validate routeMC and loop last cycle}
%\todo{validate feedsbkey last cycle}

Finally, the last round is very similar to the regime mode except that the
module $\dpStateModMC$is bypassed. In particular, the signal
$\dpStateCtrlRouteMC$ is de-asserted and the shift registers are configured to
make the data loop. No new key scheduling round is started during this last
cycle.
At the end of the last round, once the ciphertext has been fetched from the
output, a new encryption starts immediately (if $\portAESInValid$ is asserted),
or the state register is cleared by asserting the control signal
$\dpStateCtrlRouteIn$. This ensures that the core is completely clear of any
key- or plaintext-dependent data.

\end{document}
